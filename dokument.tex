%╔════════════════════════════╗
%║	  Szablon dostosował	  ║
%║	mgr inż. Dawid Kotlarski  ║
%║		  06.10.2024		  ║
%╚════════════════════════════╝
\documentclass[12pt,oneside,a4paper,openany]{article}

    % ------------------------------------------------------------------------
% PAKIETY
% ------------------------------------------------------------------------

%różne pakiety matematyczne, warto przejrzeć dokumentację, muszą być powyżej ustawień językowych.
\usepackage{mathrsfs}   %Różne symbole matematyczne opisane w katalogu ~\doc\latex\comprehensive. Zamienia \mathcal{L} ze zwykłego L na L-transformatę.
\usepackage{eucal}      %Różne symbole matematyczne.
\usepackage{amssymb}    %Różne symbole matematyczne.
\usepackage{amsmath}    %Dodatkowe funkcje matematyczne, np. polecenie \dfac{}{} skladajace ulamek w trybie wystawionym (porównaj $\dfrac{1}{2}$, a $\frac{1}{2}$).

%język polski i klawiatura
\usepackage[polish]{babel}
\usepackage{csquotes}
%\usepackage{qtimes} % czcionka Times new Roman
\usepackage{polski}

\usepackage{ifluatex}

\ifluatex
  %czcionka
  \usepackage{fontspec}
  \setmainfont{Calibri}

  %obsługa pdf'a
  \usepackage[luatex,usenames,dvipsnames]{color}      %Obsługa kolorów. Opcje usenames i dvipsnames wprowadzają dodatkowe nazwy kolorow.
  \usepackage[luatex,pagebackref=false,draft=false,pdfpagelabels=false,colorlinks=true,urlcolor=cyan,linkcolor=blue,filecolor=magenta,citecolor=green,pdfstartview=FitH,pdfstartpage=1,pdfpagemode=UseOutlines,bookmarks=true,bookmarksopen=true,bookmarksopenlevel=2,bookmarksnumbered=true,pdfauthor={Dawid Kotlarski},pdftitle={Dokumentacja Projektowa},pdfsubject={},pdfkeywords={transient recovery voltage trv},unicode=true]{hyperref}   %Opcja pagebackref=true dotyczy bibliografii: pokazuje w spisie literatury numery stron, na których odwołano się do danej pozycji.
\else
  \usepackage[pdftex,usenames,dvipsnames]{color}      %Obsługa kolorów. Opcje usenames i dvipsnames wprowadzają dodatkowe nazwy kolorow.
\usepackage[pdftex,pagebackref=false,draft=false,pdfpagelabels=false,colorlinks=true,urlcolor=blue,linkcolor=black,citecolor=green,pdfstartview=FitH,pdfstartpage=1,pdfpagemode=UseOutlines,bookmarks=true,bookmarksopen=true,bookmarksopenlevel=2,bookmarksnumbered=true,pdfauthor={Dawid Kotlarski},pdftitle={Dokumentacja Projektowa},pdfsubject={},pdfkeywords={transient recovery voltage trv},unicode=true]{hyperref}  %Opcja pagebackref=true dotyczy bibliografii: pokazuje w spisie literatury numery stron, na których odwołano się do danej pozycji.
\fi

%bibliografia
%\usepackage[numbers,sort&compress]{natbib}  %Porządkuje zawartość odnośników do literatury, np. [2-4,6]. Musi być pod pdf'em, a styl bibliogfafii musi mieć nazwę z dodatkiem 'nat', np. \bibliographystyle{unsrtnat} (w kolejności cytowania).
\usepackage[
  backend=biber,
  style=numeric,
  sorting=none
]{biblatex}
\addbibresource{bibliografia.bib}
\usepackage{hypernat}                       %Potrzebna pakietowi natbib do wspolpracy z pakietem hyperref (wazna kolejnosc: 1. hyperref, 2. natbib, 3. hypernat).

%grafika i geometria strony
\usepackage{extsizes}           %Dostepne inne rozmiary czcionek, np. 14 w poleceniu: \documentclass[14pt]{article}.
\usepackage[final]{graphicx}
\usepackage[a4paper,left=2cm,right=2cm,top=2cm,bottom=2cm]{geometry}

%strona tytułowa
\usepackage{strona_tytulowa}

%inne
\usepackage{lastpage} %! do numerowania stron w formacie (x z y)
\usepackage[hide]{todo}                     %Wprowadza polecenie \todo{treść}. Opcje pakietu: hide/show. Polecenie \todos ma byc na koncu dokumentu, wszystkie \todo{} po \todos sa ignorowane.
\usepackage[basic,physics]{circ}            %Wprowadza środowisko circuit do rysowania obwodów elektrycznych. Musi byc poniżej pakietow językowych.
\usepackage[sf,bf,outermarks]{titlesec}     %Troszczy się o wygląd tytułów rozdziałów (section, subsection, ...). sf oznacza czcionkę sans serif (typu arial), bf -- bold. U mnie: oddzielna linia dla naglowku paragraph. Patrz tez: tocloft -- lepiej robi format spisu tresci.
\usepackage{tocloft}                        %Troszczy się o format spisu trsci.
\usepackage{expdlist}    %Zmienia definicję środowiska description, daje większe możliwości wpływu na wygląd listy.
\usepackage{flafter}     %Wprowadza parametr [tb] do polecenia \suppressfloats[t] (polecenie to powoduje nie umieszczanie rysunkow, tabel itp. na stronach, na ktorych jest to polecenie (np. moze byc to stroma z tytulem rozdzialu, ktory chcemy zeby byl u samej gory, a nie np. pod rysunkiem)).
\usepackage{array}       %Ładniej drukuje tabelki (np. daje wiecej miejsca w komorkach -- nie są tak ścieśnione, jak bez tego pakietu).
\usepackage{listings}    %Listingi programow.
\usepackage[format=hang,labelsep=period,labelfont={bf,small},textfont=small]{caption}   %Formatuje podpisy pod rysunkami i tabelami. Parametr 'hang' powoduje wcięcie kolejnych linii podpisu na szerokosc nazwy podpisu, np. 'Rysunek 1.'.
\usepackage{appendix}    %Troszczy się o załączniki.
\usepackage{floatflt}    %Troszczy się o oblewanie rysunkow tekstem.
\usepackage{here}        %Wprowadza dodtkowy parametr umiejscowienia rysunków, tabel, itp.: H (duże). Umiejscawia obiekty ruchome dokladnie tam gdzie są w kodzie źródłowym dokumentu.
\usepackage{makeidx}     %Troszczy się o indeks (skorowidz).

%nieużywane, ale potencjalnie przydatne
\usepackage{sectsty}           %Formatuje nagłówki, np. żeby były kolorowe -- polecenie: \allsectionsfont{\color{Blue}}.
%\usepackage{version}           %Wersje dokumentu.

%============
\usepackage{longtable}			%tabelka
\usepackage{tabularx}
%============

%============
% Ustawienia listingów do kodu
%============

\usepackage{xcolor}

% wlasne 

% \definecolor{codegreen}{rgb}{0,0.6,0}
% \definecolor{codegray}{rgb}{0.5,0.5,0.5}
% \definecolor{codepurple}{rgb}{0.58,0,0.82}
% \definecolor{backcolour}{rgb}{0.95,0.95,0.92}

% Definicja stylu "mystyle"
% \lstdefinestyle{mystyle}{
% backgroundcolor=\color{backcolour},
% commentstyle=\color{codegreen},
% keywordstyle=\color{blue},	%magenta
% numberstyle=\tiny\color{codegray},
% stringstyle=\color{codepurple},
% basicstyle=\ttfamily\footnotesize,
% breakatwhitespace=false,
% breaklines=true,
% captionpos=b,
% keepspaces=true,
% numbers=left,
% numbersep=5pt,
% showspaces=false,
% showstringspaces=false,
% showtabs=false,
% tabsize=2,
% literate=
%   {á}{{\'a}}1 {é}{{\'e}}1 {í}{{\'i}}1 {ó}{{\'o}}1 {ú}{{\'u}}1
% {Á}{{\'A}}1 {É}{{\'E}}1 {Í}{{\'I}}1 {Ó}{{\'O}}1 {Ú}{{\'U}}1
% {à}{{\`a}}1 {è}{{\`e}}1 {ì}{{\`i}}1 {ò}{{\`o}}1 {ù}{{\`u}}1
% {À}{{\`A}}1 {È}{{\`E}}1 {Ì}{{\`I}}1 {Ò}{{\`O}}1 {Ù}{{\`U}}1
% {ä}{{\"a}}1 {ë}{{\"e}}1 {ï}{{\"i}}1 {ö}{{\"o}}1 {ü}{{\"u}}1
% {Ä}{{\"A}}1 {Ë}{{\"E}}1 {Ï}{{\"I}}1 {Ö}{{\"O}}1 {Ü}{{\"U}}1
% {â}{{\^a}}1 {ê}{{\^e}}1 {î}{{\^i}}1 {ô}{{\^o}}1 {û}{{\^u}}1
% {Â}{{\^A}}1 {Ê}{{\^E}}1 {Î}{{\^I}}1 {Ô}{{\^O}}1 {Û}{{\^U}}1
% {ã}{{\~a}}1 {ẽ}{{\~e}}1 {ĩ}{{\~i}}1 {õ}{{\~o}}1 {ũ}{{\~u}}1
% {Ã}{{\~A}}1 {Ẽ}{{\~E}}1 {Ĩ}{{\~I}}1 {Õ}{{\~O}}1 {Ũ}{{\~U}}1
% {œ}{{\oe}}1 {Œ}{{\OE}}1 {æ}{{\ae}}1 {Æ}{{\AE}}1 {ß}{{\ss}}1
% {ű}{{\H{u}}}1 {Ű}{{\H{U}}}1 {ő}{{\H{o}}}1 {Ő}{{\H{O}}}1
% {ç}{{\c c}}1 {Ç}{{\c C}}1 {ø}{{\o}}1 {Ø}{{\O}}1 {å}{{\r a}}1 {Å}{{\r A}}1
% {€}{{\euro}}1 {£}{{\pounds}}1 {«}{{\guillemotleft}}1
% {»}{{\guillemotright}}1 {ñ}{{\~n}}1 {Ñ}{{\~N}}1 {¿}{{?`}}1 {¡}{{!`}}1
% {ą}{{\k{a}}}1 {ć}{{\'{c}}}1 {ę}{{\k{e}}}1 {ł}{{\l}}1 {ń}{{\'n}}1
% {ó}{{\'o}}1 {ś}{{\'s}}1 {ź}{{\'z}}1 {ż}{{\.{z}}}1
% {Ą}{{\k{A}}}1 {Ć}{{\'{C}}}1 {Ę}{{\k{E}}}1 {Ł}{{\L}}1 {Ń}{{\'N}}1
% {Ó}{{\'O}}1 {Ś}{{\'S}}1 {Ź}{{\'Z}}1 {Ż}{{\.{Z}}}1
% }
% \lstdefinestyle{bashstyle}{
%     language=bash,
%     basicstyle=\ttfamily\footnotesize,
%     keywords={sudo, apt-get, grep, awk, sed, chmod, chown, mv, cp, rm, mkdir, rmdir, cat, ps, kill, service, systemctl, apt, },
%     keywordstyle=\color{blue!90!black}\bfseries,
%     commentstyle=\color{gray}\itshape,
%     stringstyle=\color{green!70!black},
%     numbers=left,
%     numberstyle=\raisebox{0.5ex}{\ttfamily\tiny\color{gray}}, % POPRAWIONE
%     stepnumber=1,
%     numbersep=5pt,
%     breaklines=true,
%     frame=single,
%     backgroundcolor=\color{gray!5},
%     showspaces=false,
%     showstringspaces=false,
%     showtabs=false,
%     tabsize=4,
%     captionpos=b,
%     linewidth=\linewidth,
%     xleftmargin=20pt,
%     xrightmargin=5pt,
%     framexleftmargin=15pt,
%     framexrightmargin=5pt,
%     rulecolor=\color{blue!30},
%     literate={% Polskie znaki
%       ą{{\k{a}}}1 
%       ć{{\'c}}1 
%       ę{{\k{e}}}1 
%       ł{{\l}}1 
%       ń{{\'n}}1
%       ó{{\'o}}1 
%       ś{{\'s}}1 
%       ź{{\'z}}1 
%       ż{{\.{z}}}1
%       Ą{{\k{A}}}1 
%       Ć{{\'C}}1 
%       Ę{{\k{E}}}1 
%       Ł{{\L}}1
%       Ń{{\'N}}1 
%       Ó{{\'O}}1 
%       Ś{{\'S}}1 
%       Ź{{\'Z}}1 
%       Ż{{\.{Z}}}1
%     },
%     morekeywords={if, then, else, elif, fi, for, while, do, done, case, esac, function, select, until},
%     morecomment=[l][\color{gray}\itshape]{\#},                 % Komentarze liniowe
%     morecomment=[s][\color{gray}\itshape]{\#!}{},              % Shebang POPRAWIONE
%     morestring=[b]",
%     morestring=[b]',
%     sensitive=true,
%     lineskip=2pt,
%     aboveskip=10pt,
%     belowskip=10pt,
%     numberblanklines=false
% }
% \lstset{style=mystyle} %ustawienia dla wszystkich listingów
% \lstset{style=bashstyle} %ustawienia dla wszystkich listingów
%===========

%PAGINA GÓRNA I DOLNA
\usepackage{fancyhdr}          %Dodaje naglowki jakie się chce.
\pagestyle{fancy}
\fancyhf{}
% numery stron na środku dolnej stopki
\fancyfoot[C]{\footnotesize \MakeUppercase{Bezpieczeństwo systemów informatycznych}  \\
  \normalsize\sffamily  \thepage\ z~\pageref{LastPage}}

%\fancyhead[L]{\small\sffamily \nouppercase{\leftmark}}
\fancyhead[C]{\footnotesize \textit{AKADEMIA NAUK STOSOWANYCH W NOWYM SĄCZU}\\}

\renewcommand{\headrulewidth}{0.4pt}
\renewcommand{\footrulewidth}{0.4pt}

    % ------------------------------------------------------------------------
% USTAWIENIA
% ------------------------------------------------------------------------

% ------------------------------------------------------------------------
%   Kropki po numerach sekcji, podsekcji, itd.
%   Np. 1.2. Tytuł podrozdziału
% ------------------------------------------------------------------------
\makeatletter
    \def\numberline#1{\hb@xt@\@tempdima{#1.\hfil}}                      %kropki w spisie treści
    \renewcommand*\@seccntformat[1]{\csname the#1\endcsname.\enspace}   %kropki w treści dokumentu
\makeatother

% ------------------------------------------------------------------------
%   Numeracja równań, rysunków i tabel
%   Np.: (1.2), gdzie:
%   1 - numer sekcji, 2 - numer równania, rysunku, tabeli
%   Uwaga ogólna: o otoczeniu figure ma być najpierw \caption{}, potem \label{}, inaczej odnośnik nie działa!
% ------------------------------------------------------------------------
\makeatletter
    \@addtoreset{equation}{section} %resetuje licznik po rozpoczęciu nowej sekcji
    \renewcommand{\theequation}{{\thesection}.\@arabic\c@equation} %dodaje kropki

    \@addtoreset{figure}{section}
    \renewcommand{\thefigure}{{\thesection}.\@arabic\c@figure}

    \@addtoreset{table}{section}
    \renewcommand{\thetable}{{\thesection}.\@arabic\c@table}
\makeatother

% ------------------------------------------------------------------------
% Tablica
% ------------------------------------------------------------------------
\newenvironment{tabela}[3]
{
    \begin{table}[!htb]
    \centering
    \caption[#1]{#2}
    \vskip 9pt
    #3
}{
    \end{table}
}

% ------------------------------------------------------------------------
% Dostosowanie wyglądu pozycji listy \todos, np. zamiast 'p.' jest 'str.'
% ------------------------------------------------------------------------
\renewcommand{\todoitem}[2]{%
    \item \label{todo:\thetodo}%
    \ifx#1\todomark%
        \else\textbf{#1 }%
    \fi%
    (str.~\pageref{todopage:\thetodo})\ #2}
\renewcommand{\todoname}{Do zrobienia...}
\renewcommand{\todomark}{~uzupełnić}

% ------------------------------------------------------------------------
% Definicje
% ------------------------------------------------------------------------
\def\nonumsection#1{%
    \section*{#1}%
    \addcontentsline{toc}{section}{#1}%
    }
\def\nonumsubsection#1{%
    \subsection*{#1}%
    \addcontentsline{toc}{subsection}{#1}%
    }
\reversemarginpar %umieszcza notki po lewej stronie, czyli tam gdzie jest więcej miejsca
\def\notka#1{%
    \marginpar{\footnotesize{#1}}%
    }
\def\mathcal#1{%
    \mathscr{#1}%
    }
\newcommand{\atp}{ATP/EMTP} % Inaczej: \def\atp{ATP/EMTP}

% ------------------------------------------------------------------------
% Inne
% ------------------------------------------------------------------------
\frenchspacing                      
\hyphenation{ATP/-EMTP}             %dzielenie wyrazu w danym miejscu
\setlength{\parskip}{3pt}           %odstęp pomiędzy akapitami
\linespread{1.3}                    %odstęp pomiędzy liniami (interlinia)
\setcounter{tocdepth}{4}            %uwzględnianie w spisie treści czterech poziomów sekcji
\setcounter{secnumdepth}{4}         %numerowanie do czwartego poziomu sekcji 
\titleformat{\paragraph}[hang]      %wygląd nagłówków
{\normalfont\sffamily\bfseries}{\theparagraph}{1em}{}

%komenda do łatwiejszego wstawiania zdjęć
\newcommand*{\fg}[4][\textwidth]{
    \begin{figure}[!htb]
        \begin{center}
            \includegraphics[width=#1]{#2}
            \caption{#3}
            \label{rys:#4}
        \end{center}
    \end{figure}
}

\newcommand*{\Oznacz}[2]{
\ref{#1:#2} (s. \pageref{#1:#2})
}

\newcommand*{\OznaczZdjecie}[2][Rysunek]{
#1 \Oznacz{rys}{#2}
}
    
% \newcommand*{\OznaczKod}[1]{
% \Oznacz{lst}{#1}
% }

% \newcommand*{\ListingFile}[2]{
%     \lstinputlisting[caption=#1, label={lst:#2}, language=C++]{kod/#2.txt}
% }


    %polecenia zdefiniowane w pakiecie strona_tytulowa.sty
    \title{Implementacja systemu serwerów NIS}		%...Wpisać nazwę projektu...
    \author{Arkadiusz Ryczek}		%...Wpisać imię i nazwisko autora...
    \authorI{}
    \authorII{}		%jeśli są dwie osoby w projekcie to zostawiamy:    \authorII{}
		
	\uczelnia{AKADEMIA NAUK STOSOWANYCH \\W NOWYM SĄCZU}
    \instytut{Wydział Nauk Inżynieryjnych}
    \kierunek{Katedra Informatyki}
    \praca{DOKUMENTACJA PROJEKTOWA}
    \przedmiot{BEZPIECZEŃSTWO SYSTEMÓW INFORMATYCZNYCH}
    \prowadzacy{mgr inż. Jacek Kaleta}
    \rok{2025}

%definicja składni mikrotik
\usepackage{fancyvrb}
\DefineVerbatimEnvironment{MT}{Verbatim}%
{commandchars=\+\[\],fontsize=\small,formatcom=\color{red},frame=lines,baselinestretch=1,} 
\let\mt\verb
%zakonczenie definicji składni mikrotik

\usepackage{fancyhdr}    %biblioteka do nagłówka i stopki
\begin{document}

\renewcommand{\figurename}{Rys.}    %musi byc pod \begin{document}, bo w~tym miejscu pakiet 'babel' narzuca swoje ustawienia
\renewcommand{\tablename}{Tab.}     %j.w.
\thispagestyle{empty}               %na tej stronie: brak numeru
\stronatytulowa                     %strona tytułowa tworzona przez pakiet strona_tytulowa.tex


\pagestyle{fancy}
\newpage

%formatowanie spisu treści i~nagłówków
\renewcommand{\cftbeforesecskip}{8pt}
\renewcommand{\cftsecafterpnum}{\vskip 8pt}
\renewcommand{\cftparskip}{3pt}
\renewcommand{\cfttoctitlefont}{\Large\bfseries\sffamily}
\renewcommand{\cftsecfont}{\bfseries\sffamily}
\renewcommand{\cftsubsecfont}{\sffamily}
\renewcommand{\cftsubsubsecfont}{\sffamily}
\renewcommand{\cftparafont}{\sffamily}
%koniec formatowania spisu treści i nagłówków

\tableofcontents    %spis treści
\thispagestyle{fancy}
\newpage


\newpage

%%%%%%%%%%%%%%%%%%% treść główna dokumentu %%%%%%%%%%%%%%%%%%%%%%%%%


\newpage

\section{Założenia projektowe – wymagania}
\begin{itemize}
\item Utworzyć domenę \texttt{Active Directory} w oparciu o Windows Server 2022.
\item Skonfigurować serwer jako główny kontroler domeny.
\item Dołączyć komputery klienckie do utworzonej domeny.
\item Utworzyć konta użytkowników z różnymi poziomami uprawnień:
\begin{itemize}
\item Blokada instalacji oprogramowania
\item Ograniczenia czasowe dostępu do systemu
\item Kontrola dostępu do wybranych aplikacji
\end{itemize}
\item Wdrożyć centralnie przechowywane profile użytkowników (profile roamingowe) oraz katalogi domowe.
\item Udostępnić dedykowane foldery z różnymi poziomami dostępu dla poszczególnych użytkowników/grup domenowych.
\item Wprowadzić limity przestrzeni dyskowej:
\begin{itemize}
\item Maksymalnie \textbf{50 MB} dla profilów roamingowych
\item Maksymalnie \textbf{100 MB} dla pozostałych folderów użytkowników
\end{itemize}

\item \textbf{Tworzenie maszyn wirtualnych}
\begin{itemize}
\item \textit{Wymagane maszyny:}
\begin{itemize}
\item \textbf{1×} Serwer z systemem \texttt{Windows Server 2022 Standard}
\item \textbf{2×} Komputery klienckie z systemem \texttt{Windows 10/11 Pro}
\end{itemize}
\item \textit{Wspólne ustawienia dla wszystkich VM:}
\begin{itemize}
\item Typ systemu: \texttt{Microsoft Windows} (64-bit)
\item Tryb instalacji: \texttt{Full Desktop Experience}
\item Wymagane role:

\begin{itemize}
 \item \texttt{Active Directory Domain Services} 
 \item \texttt{DHCP Server}
 \item \texttt{GPO} (Group Policy Object)
 \item \texttt{File and Storage Services}
 \item \texttt{File Server Resource Manager}
\end{itemize}
\end{itemize}
\clearpage
\item \textit{Minimalne wymagania sprzętowe:}
\begin{itemize}
\item Pamięć RAM:
\begin{itemize}
\item Serwer: \textbf{4096 MB}
\item Klienty: \textbf{2048 MB}
\end{itemize}
\item Procesor:
\begin{itemize}
\item Serwer: \textbf{2} rdzenie
\item Klienty: \textbf{2} rdzenie
\end{itemize}
\item Dysk twardy:
\begin{itemize}
\item Serwer: \textbf{50 GB} (dynamicznie alokowany)
\item Klienty: \textbf{50 GB} (dynamicznie alokowany)
\end{itemize}
\end{itemize}
\end{itemize}

\item \textbf{Konfiguracja sieci}
\begin{itemize}
\item \textit{Wymagania:}
\begin{itemize}
\item Komunikacja wewnętrzna w ramach domeny
\item Serwer musi posiadać statyczny adres IP
\end{itemize}
\item \textit{Konfiguracja kart sieciowych w VirtualBox:}
\begin{itemize}
\item Karta 2: Sieć wewnętrzna (\texttt{Internal Network}) np. \texttt{ad-network}
\end{itemize}
\item \textit{Ustawienia adresacji:}
\begin{itemize}
\item Serwer:
\begin{itemize}
\item Statyczny IP: \textbf{192.168.10.10/24}
\item DNS: Wskazujący na siebie (127.0.0.1)
\end{itemize}
\item Klienci:
\begin{itemize}
\item IP: \texttt{DHCP} ze wskazaniem na serwer DNS
\item Alternatywnie: Statyczne IP z puli \textbf{192.168.10.20-30}
\end{itemize}
\end{itemize}
\end{itemize}



\item \textbf{Zarządzanie użytkownikami i grupami}
\begin{itemize}
\item Wykorzystaj \texttt{Group Policy Management} (GPO) do:
\begin{itemize}
\item Blokady instalacji oprogramowania: \texttt{User Configuration → Policies →\\ Administrative Templates → Control Panel/Add or Remove Programs}
\item Ograniczeń czasowych: \texttt{Computer Configuration → Policies → Windows Settings → Security Settings → Local Policies/User Rights Assignment}
\item Kontroli aplikacji: \texttt{Software Restriction Policies} lub \texttt{AppLocker}
\end{itemize}
\item Do zarządzania limitami dyskowymi użyj \texttt{File Server Resource Manager (FSRM)}:
\begin{itemize}
\item Utwórz kwoty dla folderów profilów i udziałów
\end{itemize}
\end{itemize}
\end{itemize}

\newpage
\section{Opis użytych technologii}		%2
	\newpage
\section{Schemat logiczny}		%3
% (Ma zawierać aktualne nazewnictwo i adresację IP.)
	\newpage
	\lstdefinestyle{bashstyle}{
		language=bash,
		basicstyle=\ttfamily\footnotesize,
		keywordstyle=\color{blue!90!black}\bfseries,
		commentstyle=\color{gray}\itshape,
		stringstyle=\color{green!70!black},
		numbers=left,
		numberstyle=\ttfamily\tiny\color{gray}\raisebox{0.5ex},
		stepnumber=1,
		numbersep=5pt,
		breaklines=true,
		frame=single,
		backgroundcolor=\color{gray!5},
		showspaces=false,
		showstringspaces=false,
		showtabs=false,
		tabsize=4,
		captionpos=b,
		linewidth=\linewidth,
		xleftmargin=20pt,
		xrightmargin=5pt,
		framexleftmargin=15pt,
		framexrightmargin=5pt,
		rulecolor=\color{blue!30},
		literate=  % Corrected Polish character mappings
		  {ą}{{\k{a}}}1 
		  {ć}{{\'c}}1 
		  {ę}{{\k{e}}}1 
		  {ł}{{\l}}1 
		  {ń}{{\'n}}1
		  {ó}{{\'o}}1 
		  {ś}{{\'s}}1 
		  {ź}{{\'z}}1 
		  {ż}{{\.{z}}}1
		  {Ą}{{\k{A}}}1 
		  {Ć}{{\'C}}1 
		  {Ę}{{\k{E}}}1 
		  {Ł}{{\L}}1
		  {Ń}{{\'N}}1 
		  {Ó}{{\'O}}1 
		  {Ś}{{\'S}}1 
		  {Ź}{{\'Z}}1 
		  {Ż}{{\.{Z}}}1,
		emph={sudo,apt-get,grep,awk,sed,chmod,chown,mv,cp,rm,mkdir,rmdir,cat,ps,kill,service,systemctl,apt,domain,ypinit,ypserv,ypbind,ypwhich,ypcat,ypmatch,ypdomainname},
		emphstyle=\color{magenta!70!black},
		lineskip=2pt,
		aboveskip=10pt,
		belowskip=10pt,
		numberblanklines=false
	}
\lstset{style=bashstyle} % Ustawienia dla całego dokumentu
\section{Procedury instalacyjne poszczególnych usług}		%4
% Procedury instalacyjne poszczególnych usług.
% (W podpunktach zamieścić polecenia dotyczące instalacji wdrażanych usług) 
% \input{tex/5_Testy.tex}
\newpage
\section{Wnioski}	%5

Przeprowadzone wdrożenie środowiska Windows Server z Active Directory Domain Services pozwoliło na wyciągnięcie następujących wniosków:

\subsection{Realizacja celów i efektywność zarządzania}
\begin{itemize}
	\item Wszystkie założenia projektowe zostały pomyślnie zrealizowane przy użyciu Windows Server 2022 i wirtualizacji w VirtualBox.
	\item Centralne zarządzanie poprzez Active Directory i mechanizm Group Policy znacząco usprawniają administrację systemem.
	\item Profile roamingowe zapewniają spójne doświadczenie użytkownikom niezależnie od urządzenia.
\end{itemize}

\subsection{Bezpieczeństwo i zarządzanie zasobami}
\begin{itemize}
	\item Implementacja grup domenowych i zaawansowanych mechanizmów kontroli dostępu umożliwia precyzyjne zarządzanie uprawnieniami.
	\item System kwot dyskowych (50 MB dla profili roamingowych, 100 MB dla folderów użytkowników) skutecznie ogranicza wykorzystanie przestrzeni.
\end{itemize}



Windows Server 2022 z Active Directory stanowi kompletną platformę do zarządzania infrastrukturą IT. Mimo złożoności, system oferuje intuicyjne narzędzia administracyjne dla efektywnego zarządzania. Doświadczenia z projektu mogą być bezpośrednio wykorzystane w środowiskach produkcyjnych, a zidentyfikowane rozwiązania stanowią cenną bazę wiedzy.



%%%%%%%%%%%%%%%%%%% koniec treść główna dokumentu %%%%%%%%%%%%%%%%%%%%%
\newpage
% \addcontentsline{toc}{section}{Literatura}
% Modified by: Maciej Wójs  
\printbibliography[heading=bibnumbered, label=Literatura, title=Literatura]

\newpage
\hypersetup{linkcolor=black}
\renewcommand{\cftparskip}{3pt}
\clearpage
\renewcommand{\cftloftitlefont}{\Large\bfseries\sffamily}
\listoffigures
\addcontentsline{toc}{section}{Spis rysunków}
\thispagestyle{fancy}

\newpage
\renewcommand{\cftlottitlefont}{\Large\bfseries\sffamily}
\def\listtablename{Spis tabel}
\addcontentsline{toc}{section}{Spis tabel}\listoftables
\thispagestyle{fancy}

\newpage
\renewcommand{\cftlottitlefont}{\Large\bfseries\sffamily}
\renewcommand\lstlistlistingname{Spis listingów}
\addcontentsline{toc}{section}{Spis listingów}\lstlistoflistings
\thispagestyle{fancy}

%lista rzeczy do zrobienia: wypisuje na koñcu dokumentu, patrz: pakiet todo.sty
\todos
%koniec listy rzeczy do zrobienia
\end{document}

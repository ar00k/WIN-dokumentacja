\newpage

\section{Opis użytych technologii}
\label{sec:technologie}
\subsection{Windows Server 2022}
\begin{itemize}
\item \textbf{Opis:} System operacyjny serwerowy firmy Microsoft przeznaczony do zarządzania siecią, usługami katalogowymi oraz zasobami przedsiębiorstwa.
\item \textbf{Zastosowanie:} Używany jako podstawowy system dla kontrolera domeny Active Directory oraz do hostowania usług zarządzania użytkownikami i zasobami.
\item \textbf{Wersja:} 2022 Standard
\item \textbf{Typ:} System operacyjny serwerowy (64-bitowy)
\item \textbf{Wymagania sprzętowe:}
\begin{itemize}
\item Procesor: 1.4 GHz (64-bit), 2+ rdzenie
\item Pamięć RAM: 512 MB (minimum), zalecane 4 GB+
\item Dysk: 32 GB wolnego miejsca (dynamicznie alokowane)
\end{itemize}
\end{itemize}

\subsection{Active Directory Domain Services (AD DS)}
\begin{itemize}
\item \textbf{Opis:} Usługa katalogowa Microsoftu umożliwiająca centralne zarządzanie użytkownikami, komputerami, grupami i politykami bezpieczeństwa w domenie.
\item \textbf{Zastosowanie:} Utworzenie domeny, konfiguracja kontrolera domeny (PDC), dołączanie klientów oraz zarządzanie kontami użytkowników.
\item \textbf{Wersja:} Zintegrowana z Windows Server 2022
\item \textbf{Kluczowe funkcje:}
\begin{itemize}
\item Uwierzytelnianie Kerberos
\item Replikacja danych między kontrolerami domeny
\item Zarządzanie obiektami za pomocą konsoli Active Directory Users and Computers
\end{itemize}
\end{itemize}

\subsection{Group Policy (GPO)}
\begin{itemize}
\item \textbf{Opis:} Mechanizm definiowania i wymuszania zasad konfiguracji dla użytkowników i komputerów w domenie.
\item \textbf{Zastosowanie:}
\begin{itemize}
\item Blokada instalacji oprogramowania (np. przez Software Restriction Policies)
\item Ograniczenia czasowe logowania (np. Logon Hours)
\item Kontrola dostępu do aplikacji (AppLocker)
\end{itemize}
\item \textbf{Wersja:} Zależna od funkcjonalności Windows Server 2022
\item \textbf{Typ:} Narzędzie zarządzania oparte na szablonach administracyjnych.
\end{itemize}

\subsection{Roaming User Profiles}
\begin{itemize}
\item \textbf{Opis:} Profile użytkowników przechowywane centralnie na serwerze, synchronizowane między wszystkimi komputerami w domenie.
\item \textbf{Zastosowanie:} Utworzenie profilu przenośnego oraz katalogu macierzystego (home directory) na zasobie sieciowym.
\item \textbf{Wymagania:}
\begin{itemize}
\item Udział sieciowy z uprawnieniami zapisu dla użytkowników
\item Konfiguracja ścieżki profilu w właściwościach konta użytkownika w AD
\end{itemize}
\end{itemize}

\subsection{NTFS i Udziały Sieciowe (SMB)}
\begin{itemize}
\item \textbf{Opis:} System plików NTFS oraz protokół SMB umożliwiający udostępnianie zasobów w sieci.
\item \textbf{Zastosowanie:}
\begin{itemize}
\item Udostępnianie folderów z określonymi uprawnieniami (np. tylko dla wybranych grup w domenie)
\item Kontrola dostępu poprzez kombinację uprawnień NTFS i udziałów sieciowych
\end{itemize}
\item \textbf{Wersja:} NTFS 3.1, SMB 3.1.1 (Windows Server 2022)
\item \textbf{Bezpieczeństwo:} Szyfrowanie transmisji danych (SMB Encryption).
\end{itemize}

\subsection{Dyskowe Kwoty (Disk Quotas)}
\begin{itemize}
\item \textbf{Opis:} Funkcja systemu Windows Server ograniczająca przestrzeń dyskową przydzieloną użytkownikom.
\item \textbf{Zastosowanie:}
\begin{itemize}
\item Ograniczenie profilu przechodniego do 50 MB
\item Limit 100 MB dla innych folderów użytkowników
\end{itemize}
\item \textbf{Konfiguracja:} Narzędzie File Server Resource Manager (FSRM) lub edycja właściwości partycji NTFS.
\end{itemize}

\subsection{Windows 10/11 Enterprise}
\begin{itemize}
\item \textbf{Opis:} System operacyjny dla klientów domeny z obsługą zaawansowanych funkcji enterprise.
\item \textbf{Zastosowanie:} Komputery dołączone do domeny Active Directory w celu testowania polityk i ograniczeń.
\item \textbf{Wymagania:}
\begin{itemize}
\item Wersja Pro lub Enterprise
\item Połączenie sieciowe z kontrolerem domeny
\end{itemize}
\end{itemize}

\subsection{VirtualBox}
\begin{itemize}
\item \textbf{Opis:} Platforma wirtualizacyjna do emulacji środowisk sieciowych.
\item \textbf{Zastosowanie:} Uruchomienie maszyn wirtualnych z Windows Server 2022 i klientami.
\item \textbf{Wersja:} 7.0+
\item \textbf{Konfiguracja sieci:} Tryb Internal Network lub Bridged Adapter dla symulacji domeny.
\end{itemize}
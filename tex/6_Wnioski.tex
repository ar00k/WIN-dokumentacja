\newpage
\section{Wnioski}	%5

Przeprowadzone wdrożenie środowiska Windows Server z Active Directory Domain Services pozwoliło na wyciągnięcie następujących wniosków:

\subsection{Realizacja celów i efektywność zarządzania}
\begin{itemize}
	\item Wszystkie założenia projektowe zostały pomyślnie zrealizowane przy użyciu Windows Server 2022 i wirtualizacji w VirtualBox.
	\item Centralne zarządzanie poprzez Active Directory i mechanizm Group Policy znacząco usprawniają administrację systemem.
	\item Profile roamingowe zapewniają spójne doświadczenie użytkownikom niezależnie od urządzenia.
\end{itemize}

\subsection{Bezpieczeństwo i zarządzanie zasobami}
\begin{itemize}
	\item Implementacja grup domenowych i zaawansowanych mechanizmów kontroli dostępu umożliwia precyzyjne zarządzanie uprawnieniami.
	\item System kwot dyskowych (50 MB dla profili roamingowych, 100 MB dla folderów użytkowników) skutecznie ogranicza wykorzystanie przestrzeni.
\end{itemize}



Windows Server 2022 z Active Directory stanowi kompletną platformę do zarządzania infrastrukturą IT. Mimo złożoności, system oferuje intuicyjne narzędzia administracyjne dla efektywnego zarządzania. Doświadczenia z projektu mogą być bezpośrednio wykorzystane w środowiskach produkcyjnych, a zidentyfikowane rozwiązania stanowią cenną bazę wiedzy.

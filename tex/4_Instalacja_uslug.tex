\newpage
	\lstdefinestyle{bashstyle}{
		language=bash,
		basicstyle=\ttfamily\footnotesize,
		keywordstyle=\color{blue!90!black}\bfseries,
		commentstyle=\color{gray}\itshape,
		stringstyle=\color{green!70!black},
		numbers=left,
		numberstyle=\ttfamily\tiny\color{gray}\raisebox{0.5ex},
		stepnumber=1,
		numbersep=5pt,
		breaklines=true,
		frame=single,
		backgroundcolor=\color{gray!5},
		showspaces=false,
		showstringspaces=false,
		showtabs=false,
		tabsize=4,
		captionpos=b,
		linewidth=\linewidth,
		xleftmargin=20pt,
		xrightmargin=5pt,
		framexleftmargin=15pt,
		framexrightmargin=5pt,
		rulecolor=\color{blue!30},
		literate=  % Corrected Polish character mappings
		  {ą}{{\k{a}}}1 
		  {ć}{{\'c}}1 
		  {ę}{{\k{e}}}1 
		  {ł}{{\l}}1 
		  {ń}{{\'n}}1
		  {ó}{{\'o}}1 
		  {ś}{{\'s}}1 
		  {ź}{{\'z}}1 
		  {ż}{{\.{z}}}1
		  {Ą}{{\k{A}}}1 
		  {Ć}{{\'C}}1 
		  {Ę}{{\k{E}}}1 
		  {Ł}{{\L}}1
		  {Ń}{{\'N}}1 
		  {Ó}{{\'O}}1 
		  {Ś}{{\'S}}1 
		  {Ź}{{\'Z}}1 
		  {Ż}{{\.{Z}}}1,
		emph={sudo,apt-get,grep,awk,sed,chmod,chown,mv,cp,rm,mkdir,rmdir,cat,ps,kill,service,systemctl,apt,domain,ypinit,ypserv,ypbind,ypwhich,ypcat,ypmatch,ypdomainname},
		emphstyle=\color{magenta!70!black},
		lineskip=2pt,
		aboveskip=10pt,
		belowskip=10pt,
		numberblanklines=false
	}
\lstset{style=bashstyle} % Ustawienia dla całego dokumentu
\section{Procedury instalacyjne poszczególnych usług}		%4
% Procedury instalacyjne poszczególnych usług.
% (W podpunktach zamieścić polecenia dotyczące instalacji wdrażanych usług)

\subsection{Instalacja systemu operacyjnego Windows Server 2022}


\fg[0.65\linewidth]{rys/Maszyny/01.png}{Konfiguracja maszyny wirtualnej w VirtualBox: wybór nazwy, lokalizacji, typu systemu (Windows Server 2022) oraz alokacji zasobów (RAM: 4096 MB, dysk: 50 GB).}{Maszyny-01}

\fg[0.65\linewidth]{rys/Maszyny/02.png}{Wybranie rodzaju karty sieciowej}{Maszyny-02}
\clearpage

\fg[0.65\linewidth]{rys/Maszyny/03.png}{Ekran wyboru wersji systemu Windows Server 2022 (Standard Edition) podczas instalacji.}{Maszyny-03}
\fg[0.65\linewidth]{rys/Maszyny/04.png}{Ustawianie hasła}{Maszyny-04}
\clearpage

\fg[0.65\linewidth]{rys/Maszyny/05.png}{Podsumowanie konfiguracji maszyny wirtualnej przed uruchomieniem instalacji.}{Maszyny-05}
\fg[0.65\linewidth]{rys/Maszyny/06.png}{Wybór edycji systemu Windows 10 pro z listy dostępnych opcji.}{Maszyny-06}
\clearpage

\subsection{Instalacja usługi Active Directory}

\fg[0.65\linewidth]{rys/AD/01.png}{Dodawanie roli Active Directory Domain Services w Server Manager – wybór wymaganych komponentów.}{AD-01}
\fg[0.65\linewidth]{rys/AD/01.png}{Potwierdzenie instalacji narzędzi administracyjnych AD DS.}{AD-01}
\clearpage

\fg[0.65\linewidth]{rys/AD/02.png}{Ekran potwierdzenia wybranych ról do instalacji (AD DS).}{AD-02}
\fg[0.65\linewidth]{rys/AD/03.png}{Komunikat o konieczności dokończenia konfiguracji AD DS po restarcie serwera.}{AD-03}
\clearpage

\fg[0.65\linewidth]{rys/AD/04.png}{Konfiguracja nowego lasu Active Directory – wprowadzenie nazwy domeny ar.ad.}{AD-04}
\fg[0.65\linewidth]{rys/AD/05.png}{Ustawienie poziomu funkcjonalności lasu i domeny (Windows Server 2016)}{AD-05}
\clearpage

\fg[0.65\linewidth]{rys/AD/06.png}{Komunikat o pomyślnej promocji serwera na kontroler domeny.}{AD-06}
\fg[0.65\linewidth]{rys/AD/07.png}{Ekran logowania do nowo utworzonej domeny AR po restarcie.}{AD-07}
\clearpage

\fg[0.65\linewidth]{rys/AD/08.png}{Zmiana członkostwa komputera w domenie (System → Advanced system settings).}{AD-08}
\fg[0.65\linewidth]{rys/AD/09.png}{Widok logowania użytkownika domenowego z uwzględnieniem nazwy domeny.}{AD-09}
\clearpage

\subsection{Instalacja usługi DHCP}

\fg[0.65\linewidth]{rys/DHCP/01.png}{Dodawanie roli DHCP Server z wymaganymi narzędziami administracyjnymi.}{DHCP-01}
\fg[0.65\linewidth]{rys/DHCP/02.png}{Potwierdzenie instalacji DHCP Server wraz z komponentami.}{DHCP-02}
\clearpage

\fg[0.65\linewidth]{rys/DHCP/03.png}{Kontynuacja konfiguracji DHCP.}{DHCP-03}
\fg[0.65\linewidth]{rys/DHCP/04.png}{Wybór użytkownika który zarządza DHCP}{DHCP-04}
\clearpage

\fg[0.65\linewidth]{rys/DHCP/05.png}{Przejscie do DHCP Manager.}{DHCP-05}
\fg[0.65\linewidth]{rys/DHCP/06.png}{Dodawanie nowego zakresu.}{DHCP-06}
\clearpage

\fg[0.65\linewidth]{rys/DHCP/07.png}{Nazwa nowego zarkesu.}{DHCP-07}
\fg[0.65\linewidth]{rys/DHCP/08.png}{Przypisanie początku i końca zakresu.}{DHCP-08}
\clearpage

\fg[0.65\linewidth]{rys/DHCP/09.png}{Ustawienie bramy.}{DHCP-09}
\fg[0.65\linewidth]{rys/DHCP/10.png}{Ustawienie DNS.}{DHCP-10}
\clearpage

\fg[0.65\linewidth]{rys/DHCP/11.png}{Finalizacja DHCP.}{DHCP-11}

\subsection{Użytkownicy i grupy AD}

\fg[0.65\linewidth]{rys/uzytkownicyAD/01.png}{Widok konsoli Active Directory Users and Computers z listą obiektów domeny.}{uzytkownicyAD-01}
\fg[0.65\linewidth]{rys/uzytkownicyAD/02.png}{Menu kontekstowe do tworzenia nowych użytkowników i grup w domenie.}{uzytkownicyAD-02}
\clearpage

\fg[0.65\linewidth]{rys/uzytkownicyAD/03.png}{Dodanie 2 komputerów.}{uzytkownicyAD-03}
\fg[0.65\linewidth]{rys/uzytkownicyAD/04.png}{Dodawanie nowej grupy}{uzytkownicyAD-04}
\clearpage

\fg[0.65\linewidth]{rys/uzytkownicyAD/05.png}{Dodawanie grupy pracownicy.}{uzytkownicyAD-05}
\fg[0.65\linewidth]{rys/uzytkownicyAD/06.png}{Dodawanie grupy IT.}{uzytkownicyAD-06}
\clearpage

\fg[0.65\linewidth]{rys/uzytkownicyAD/07.png}{Dodawanie użytkownika.}{uzytkownicyAD-07}
\fg[0.65\linewidth]{rys/uzytkownicyAD/08.png}{Dodawanie użytkownika Arkadiusz Ryczek.}{uzytkownicyAD-08}
\clearpage

\subsection{Instalacja usługi GPO}

\fg[0.65\linewidth]{rys/GPO/01.png}{Tworzenie nowego policy.}{GPO-01}
\fg[0.65\linewidth]{rys/GPO/02.png}{Edycja policy.}{GPO-02}
\clearpage

\fg[0.65\linewidth]{rys/GPO/03.png}{Wyłączanie opcji instalowania oprogramowania}{GPO-03}
\fg[0.65\linewidth]{rys/GPO/04.png}{Wyłączanie opcji instalowania oprogramowania cz2}{GPO-04}
\clearpage



\subsection{Profil centralny i katalog macierzysty}

\fg[0.65\linewidth]{rys/katalogi/01.png}{Udostępnianie folderu Domowy z uprawnieniami dla grup Pracownicy i IT.}{katalogi-01}
\fg[0.65\linewidth]{rys/katalogi/02.png}{Udostępnianie folderu Domowy z uprawnieniami dla grup Pracownicy i IT cz2.}{katalogi-02}
\clearpage


\fg[0.65\linewidth]{rys/katalogi/03.png}{Udostepnianie katalogi Profile}{katalogi-03}
\fg[0.65\linewidth]{rys/katalogi/04.png}{Dodanie ścieżek dla użytkowników}{katalogi-04}
\clearpage


\subsection{Konfiguracja kwot dyskowych}

\fg[0.65\linewidth]{rys/Quota/01.png}{Instalacja usługi File Server Resource Manager}{Quota-01}
\fg[0.65\linewidth]{rys/Quota/02.png}{Tworzenie nowego szablonu kwoty.}{Quota-02}
\clearpage

\fg[0.55\linewidth]{rys/Quota/03.png}{Ustawianie limitów.}{Quota-03}
\fg[0.65\linewidth]{rys/Quota/04.png}{Tworzenie nowej kwoty.}{Quota-04}

\fg[0.55\linewidth]{rys/Quota/05.png}{Ustawienie kwoty dla \texttt{C:\textbackslash Domowy}}{Quota-05}
\fg[0.55\linewidth]{rys/Quota/06.png}{Ustawienie kwoty dla \texttt{C:\textbackslash Profile}}{Quota-06}
\clearpage

\clearpage
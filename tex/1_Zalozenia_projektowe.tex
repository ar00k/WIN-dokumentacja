\newpage

\section{Założenia projektowe – wymagania}
\begin{itemize}
\item Utworzyć domenę \texttt{Active Directory} w oparciu o Windows Server 2022.
\item Skonfigurować serwer jako główny kontroler domeny.
\item Dołączyć komputery klienckie do utworzonej domeny.
\item Utworzyć konta użytkowników z różnymi poziomami uprawnień:
\begin{itemize}
\item Blokada instalacji oprogramowania
\item Ograniczenia czasowe dostępu do systemu
\item Kontrola dostępu do wybranych aplikacji
\end{itemize}
\item Wdrożyć centralnie przechowywane profile użytkowników (profile roamingowe) oraz katalogi domowe.
\item Udostępnić dedykowane foldery z różnymi poziomami dostępu dla poszczególnych użytkowników/grup domenowych.
\item Wprowadzić limity przestrzeni dyskowej:
\begin{itemize}
\item Maksymalnie \textbf{50 MB} dla profilów roamingowych
\item Maksymalnie \textbf{100 MB} dla pozostałych folderów użytkowników
\end{itemize}

\item \textbf{Tworzenie maszyn wirtualnych}
\begin{itemize}
\item \textit{Wymagane maszyny:}
\begin{itemize}
\item \textbf{1×} Serwer z systemem \texttt{Windows Server 2022 Standard}
\item \textbf{2×} Komputery klienckie z systemem \texttt{Windows 10/11 Pro}
\end{itemize}
\item \textit{Wspólne ustawienia dla wszystkich VM:}
\begin{itemize}
\item Typ systemu: \texttt{Microsoft Windows} (64-bit)
\item Tryb instalacji: \texttt{Full Desktop Experience}
\item Wymagane role:

\begin{itemize}
 \item \texttt{Active Directory Domain Services} 
 \item \texttt{DHCP Server}
 \item \texttt{GPO} (Group Policy Object)
 \item \texttt{File and Storage Services}
 \item \texttt{File Server Resource Manager}
\end{itemize}
\end{itemize}
\clearpage
\item \textit{Minimalne wymagania sprzętowe:}
\begin{itemize}
\item Pamięć RAM:
\begin{itemize}
\item Serwer: \textbf{4096 MB}
\item Klienty: \textbf{2048 MB}
\end{itemize}
\item Procesor:
\begin{itemize}
\item Serwer: \textbf{2} rdzenie
\item Klienty: \textbf{2} rdzenie
\end{itemize}
\item Dysk twardy:
\begin{itemize}
\item Serwer: \textbf{50 GB} (dynamicznie alokowany)
\item Klienty: \textbf{50 GB} (dynamicznie alokowany)
\end{itemize}
\end{itemize}
\end{itemize}

\item \textbf{Konfiguracja sieci}
\begin{itemize}
\item \textit{Wymagania:}
\begin{itemize}
\item Komunikacja wewnętrzna w ramach domeny
\item Serwer musi posiadać statyczny adres IP
\end{itemize}
\item \textit{Konfiguracja kart sieciowych w VirtualBox:}
\begin{itemize}
\item Karta 2: Sieć wewnętrzna (\texttt{Internal Network}) np. \texttt{ad-network}
\end{itemize}
\item \textit{Ustawienia adresacji:}
\begin{itemize}
\item Serwer:
\begin{itemize}
\item Statyczny IP: \textbf{192.168.10.10/24}
\item DNS: Wskazujący na siebie (127.0.0.1)
\end{itemize}
\item Klienci:
\begin{itemize}
\item IP: \texttt{DHCP} ze wskazaniem na serwer DNS
\item Alternatywnie: Statyczne IP z puli \textbf{192.168.10.20-30}
\end{itemize}
\end{itemize}
\end{itemize}



\item \textbf{Zarządzanie użytkownikami i grupami}
\begin{itemize}
\item Wykorzystaj \texttt{Group Policy Management} (GPO) do:
\begin{itemize}
\item Blokady instalacji oprogramowania: \texttt{User Configuration → Policies →\\ Administrative Templates → Control Panel/Add or Remove Programs}
\item Ograniczeń czasowych: \texttt{Computer Configuration → Policies → Windows Settings → Security Settings → Local Policies/User Rights Assignment}
\item Kontroli aplikacji: \texttt{Software Restriction Policies} lub \texttt{AppLocker}
\end{itemize}
\item Do zarządzania limitami dyskowymi użyj \texttt{File Server Resource Manager (FSRM)}:
\begin{itemize}
\item Utwórz kwoty dla folderów profilów i udziałów
\end{itemize}
\end{itemize}
\end{itemize}
